%%%%%%%%%%%%%%%%%%%%%%%%%%%%%%%%%%%%%%%%%%%%%%%%%%%%%%%%%%%%%%%%%%%%%%%%%%%%%%%%
%%%%                           Tarvo Reinpalu                               %%%%
%%%% Maximizing UITableView scrolling performance in Pairby iOS Application %%%%
%%%%                             20.02.2017                                 %%%%
%%%%%%%%%%%%%%%%%%%%%%%%%%%%%%%%%%%%%%%%%%%%%%%%%%%%%%%%%%%%%%%%%%%%%%%%%%%%%%%%

%%%%%%%%%%%%%%%%%%%%%%%%%
% General configuration %
%%%%%%%%%%%%%%%%%%%%%%%%%
\documentclass[a4paper,12pt]{article}
\usepackage[utf8]{inputenc}
\usepackage[T1]{fontenc}
\usepackage[english]{babel}
\usepackage{itk_thesis_eng}
\usepackage{cite}
\usepackage{mathptmx} % Apply Times New Roman font
\usepackage[toc, xindy, style=list, nonumberlist]{glossaries}
\usepackage{url}
\graphicspath{{img/}}
\linespread{1.5}
\usepackage{hyperref}
\usepackage{tabularx} % Better tables (fill width)

\usepackage{minted} % Code coloring
\usepackage{xcolor}
\usemintedstyle{vs}
\definecolor{codebg}{gray}{0.96}
\setminted{
  bgcolor=codebg,
  tabsize=2,
  fontsize=\scriptsize,
  breaklines=true,
  escapeinside=\#\#}

\begin{document}

%%%%%%%%%%%%%%%%%%%%
% Thesis constants %
%%%%%%%%%%%%%%%%%%%%
\begin{itkTitlePage}
\title{Optimizing UITableView in the Pairby iOS application}
\paper{Diploma Thesis}
\author{Tarvo Reinpalu}
\curriculum{IT systems development curriculum}
\supervisor{Gary Planthaber}
\consultant{Toomas Lepikult}
\end{itkTitlePage}



%%%%%%%%%%%%%%%%%%%%%%%%%%%%%%
%%%% Pre-content chapters %%%%
%%%%%%%%%%%%%%%%%%%%%%%%%%%%%%

%%%%%%%%%%%%%%%%%%%%%%
% Author declaration %
%%%%%%%%%%%%%%%%%%%%%%
\itkMakeAuthorDeclaration

%%%%%%%%%%%%%%%%%%%%%
% Table of contents %
%%%%%%%%%%%%%%%%%%%%%
\clearpage
\thispagestyle{empty}
\setcounter{tocdepth}{2}
\tableofcontents



%%%%%%%%%%%%%%%%%%%%%%%%%%
%%%% Content chapters %%%%
%%%%%%%%%%%%%%%%%%%%%%%%%%

%%%%%%%%%%%%%%%% %%%%%%%%%%%%%%%%%%%%%%%%%%
% Introduction % % REVIEWED | T: + | G:   %
%%%%%%%%%%%%%%%% %%%%%%%%%%%%%%%%%%%%%%%%%%
\iosection{Introduction}
Mobile applications' user interface responsiveness is necessary to keep the brand's reputation and keep the user using the application, as shown in online studies.\cite{AppSpeedStudyHP}\cite{AppSpeedStudyApigee} 55\% of the people who participated in a research by HP said that only the application is to blame for bad performance and 37\% stated that errors and freezes made them think less of a company’s brand.\cite{AppSpeedStudyHP} A study by Apigee revealed that 18\% of the survey participants are likely to delete the application after it freezes for more than 5 seconds.\cite{AppSpeedStudyApigee} The percentage of the Apigee survey participants who would delete the application became 38\% when the freeze time would exceed 30 seconds.\cite{AppSpeedStudyApigee} The Apigee survey also revealed that two of five most common reasons for bad reviews were related to user interface responsiveness: freezes were the reason for leaving a bad review for 76\% of the survey participants and slow responsiveness for 59\% of the participants.\cite{AppSpeedStudyApigee}

The expectation of having an instantaneously responding user interface has been around as long as user interfaces.\cite{NielsenUsabilityEngineering} Jakob Nielsen, a web usability consultant who holds a PH.d in human–computer interaction\cite{JakobNielsenNNGroup}, has written in his literature that even the slightest delay (anything above 0.1 seconds) is not considered instantaneous by users and anything above 1 second results in the user's flow of thought to be interrupted and focus to be lost.\cite{NielsenUsabilityEngineering} Anything above 10 seconds means the user's attention is completely lost and without any feedback about progress, they might not return their attention to the application.\cite{NielsenUsabilityEngineering}

The researches cited in the previous two paragraphs show that users expect a responsive user interface and if the application fails to provide it, the users might delete the application, leave a bad review of the application or both.

Pairby, a mobile dating service of which the author is a co-founder of, has applications on mobile platforms Android and iOS and none of the two Pairby applications are an exception to the users' previously described expectation of having a responsive user interface. This thesis' subject is directly derived from the author's interest in solving a user interface responsiveness and low frame rate issue in a Pairby iOS application's vertically scrollable view, which uses the iOS UITableView component, with the goal of avoiding both losing users and bad reviews.

The following subsections will describe the problem and declare the goal of the thesis in greater detail and describe how the thesis is structured and how the problem was approached.

\iosubsection{Description of the problem}
Pairby's iOS application contains a multitude of different vertically scrolling views, which can contain a large number of subviews with no defined limit. One of the mentioned vertically scrolling views is the message list view, which this thesis focuses on. The message list view uses a UITableView to render a vertical list of views (called cells) that display messages, which can contain either only text or one or multiple visual media items.

Pairby iOS application's first prototype encountered user interface responsiveness and low frame rate issues in the message list view. The prototype project was discarded, in part, because of the user interface responsiveness and low frame rate issues. Exploring ways to improve on the two encountered issues in the message list view, and other similar scrollable views, was something that needed to be done before building the new project.

At the time of writing this thesis, the development of the new iOS application for Pairby had started, but a solution to the user interface responsiveness and low frame rate issues in the vertically scrollable message list view had not yet been found.

\iosubsection{Goal of the thesis}
The goal of the thesis was to explore, analyze and document ways to improve the Pairby iOS application's vertically scrolling views' user interface responsiveness and frame rate. More specifically, find ways to improve the UITableView component's performance in the two previously mentioned performance aspects when rendering complex views that contain either text or one or multiple visual media items.

The optimizations were expected to have a positive impact on user interface responsiveness or frame rate or both and be applicable to the Pairby iOS application's vertically scrollable views, including the message list view. All the optimizations were expected to be documented as independently of each other as possible, to allow for their use as separate resources.

\iosubsection{Structure of the thesis}
Reaching the goal of this thesis required a strictly structured approach. The following paragraphs will describe the order of operations chosen to approach the problem.

\paragraph*{Specifications}
Specifications section focused on producing a set specifications describing what content the message objects could contain and how the message views should look like.

\paragraph*{Measuring performance}
The "measuring performance" section explains the techniques used to measure different user interface responsiveness and device performance related metrics, explains in detail how the test environment was created, how and on what device the tests are run, and how the metrics are analyzed.

\paragraph*{Comparison points}
In the comparison points section, two different vertically scrollable views were created, utilizing both the specification and the set of sample message objects produced. The two scrollable views were named "best case scenario" and "worst case scenario", where the first one was purpose built to display the best possible performance of the device it was run on and the latter one to display the worst performance.

\paragraph*{Performance optimizations}
The performance optimizations part of the thesis focuses on exploring and analyzing different optimization methods. All the found performance optimizations were built into the same project, with the starting point being the worst case scenario, and benchmarked after each optimization.

\paragraph*{Findings}
The findings section compares the combined result of all optimizations against the two reference points called best and worst case scenario and analyzes the results.

%%%%%%%%%%%%%%%%%% %%%%%%%%%%%%%%%%%%%%%%%%%%
% Specifications % % REVIEWED | T: + | G:   %
%%%%%%%%%%%%%%%%%% %%%%%%%%%%%%%%%%%%%%%%%%%%
\section{Specifications}
\label{sec:specifications}
The test project was designed to broadly mimic the user interface and data layer of the production Pairby mobile applications message list view. The following subsections describe the requirements for the test data and user interface as well as how the test data was generated.

\subsection{Test data requirements}
The test data contained 500 message type objects. The data was stored in JSON format and saved to a file, which could then be loaded into the iOS test project using the XCode assets interface.

\subsubsection{Message type}
The type definition for each message object is shown below, in \autoref{tbl:message-type-definition}.

\vskip.2in

\begin{table}[H]
  \caption{Message type definition}
  \label{tbl:message-type-definition}
  \begin{tabularx}{\textwidth}{| l | l | l | X |}
    \hline
    Property & Type & Required & Description \\
    \hline
    messageId & Integer & true & A unique identifier generated from a sequence starting at 0 with an increment of 1. \\
    direction & String & true & Can be either "in" or "out", representing the direction of the message relative to the current user. \\
    message & String & false & Text message's text. Not present in media messages. \\
    mediaItems & String[] & false & Media message's media URLs. Not present in text messages. \\
    \hline
  \end{tabularx}
\end{table}

\subsubsection{Text message object}
\label{subsec:text-message-object}
Text messages are messages that contain only text. The content chosen for the generated text messages is tightly tied to the way they are expected to be rendered in the user interface. To get a set of message views that have a variable height and width, with the goal of simulating real-life use cases, the number of lines in the text and the number of characters per line had to be taken into consideration.

The width of the text message view in the user interface is dependant on the length of the longest line of text in the text message object's content. That dependency was taken into consideration when generating text messages' content. For the purposes of simulating real-life scenarios, text messages of three different widths were generated.

The height of the text message view in the user interface is dependent on the number of lines in the text message object's content. That dependency was also taken into consideration when generating text messages' content. To broadly simulate real-life scenarios, text messages of three different heights were generated. The number of lines was controlled using line break characters.

Combining the two dependencies and user interface constraints mentioned in the previous two paragraphs resulted in three text message contents being created. Each of the three text message contents were expected to have different heights and widths.

\subsubsection{Media message object}
\label{subsec:media-message-object}
Media messages are messages that contain only media items. In this thesis' project, they can contain up to three visual media items as opposed to the maximum of 6 visual media items or 1 audio file in the production Pairby applications. The lower number and variety of media items was chosen to simplify and speed up the building of the thesis project.

The visual media items, or images, can either be static (in JPG format) or animated (in GIF format). To simulate real-life conditions, all media was expected to be loaded from the Pairby servers. This also raised the requirements of running all the tests on the same network to guarantee similar environmental conditions for all test cases.

\subsection{Test data generation}
\label{subsec:test-data-generation}
Test data for the project was generated in a separate script, which was written in JavaScript and executed in a Node.JS environment. The output of the script was a JSON file that contained an array of message type objects. The following subsections describe in detail how the message generation script worked.

\subsubsection{Message object generation}
The number of messages to be generated was configurable and was set to 500 for the purposes of this project.

The type (text or media) of each message was decided using a semi-random probability technique: a random number from 0 to 1 was generated and compared against a configurable parameter called "media message probability". If the generated number was less than or equal to the media message probability parameter, the message type was set to be media message and therefore its mediaItems value was then defined. If the generated number was larger than the media message probability parameter, then the message was set to be of text type and its message value was then defined. The test data set was generated with the media message probability set to 0.25 (25\%).

Each messageId value was simply set to the sequence number in the running loop (which started at zero and ended at the number of generated messages minus 1). For the 500 messages generated for the purpose of this test, the values were in the range of 0 to 499 (inclusive).

Each message's direction value was also defined through a random number generator and was set to have an equal chance of either being outbound or inbound. This process was similar to the way the message type was chosen: a random number between 0 and 1 was generated and it was compared against 0.5, if was less than or equal to 0.5, direction value was set to "out", otherwise it was set to "in". This random number generation technique meant in theory that half of the messages would be outbound and the other half inbound, but not in a pre-determined order.

\subsubsection{Text message content generation}
As defined in \autoref{subsec:text-message-object}, text messages of three different heights and widths exist. The text for each different size of text message was static (pre-defined). Selecting one of the three different contents (which would determine the size of the message view) was random and each size had an equal chance of being selected. A random whole number between 1 and 3 (inclusive) was generated. If the generated number was 1, then the smallest width and height message content was added to the message object. If the generated number was 2, then the medium size message content was added to the message object. If the number was 3, then the tallest and widest message content was added to the message object.

\subsubsection{Media message content generation}
As defined in \autoref{subsec:media-message-object}, media messages can contain up to three visual media items (images) which were expected to be loaded from the Pairby servers, through the Internet. The constraint of containing either one, two or three images limited the number of different media message view types to three. The three media types were media message views that contain either one, two or three images.

The set of sample images consisted of 100 static images of JPG format and 10 animated images of GIF format. The data generator script uploaded the images to the Pairby servers and then returned an array of URLs that point to the image files on the Pairby server.

The number of visual media items to include with a specific media message was a random number between 1 and 3 (inclusive). The image URL, or URLs, that point to the images were chosen randomly and duplicates were not prevented. The lack of duplicate prevention meant that there could be media messages that contained the same media item URL multiple times. Even though this would not be possible in the production Pairby application, there was no direct reason to prevent it from happening in the thesis project as it should still work as expected.

\subsection{User interface requirements}
As stated in the introduction paragraph of \autoref{sec:specifications}, the user interface was designed to mimic the Pairby applications' message list view. The following subsections describe in detail what the message views in the messaging list view were expected to look like.

\subsubsection{Common design elements}
Both the text and media message views were expected to have rounded corners with a radius of 15 points and a border of gray colour (\mintinline{apacheconf}{#ECEFF1}) with a width of 1 point.

\subsubsection{Text message view}
The text message view was expected to handle line breaks, which means it should expand vertically to accommodate multiple lines of text. The background color for inbound and outbound messages was expected to be different: teal (\mintinline{apacheconf}{#4DB6AC}) for outbound, gray (\mintinline{apacheconf}{#ECEFF1}) for inbound.

\subsubsection{Media message view}
\label{subsec:media-message-view}
The media message view stylings are dependant on the number of images they contain, as is the case in the production Pairby applications. The following paragraphs define the layout and size of the three possible view types, which were media message views containing either one, two or three media items.

\paragraph*{1 media item}
A single media item message should display the image with a width and height of 180 points.

\paragraph*{2 media items}
A two media item message should display the images side by side, with the image widths and heights being 90 points. The gap between the media items should be 10 points.

\paragraph*{3 media items}
A three media item message should display the images side by side, with the image widths and heights being 90 points. The gap between the media items should be 10 points.

\autoref{ui-mockup}, shown below, displays the three different media message view types and their size, both in points and in relation to each other.
\itkIncludeImage{ui-mockup}{User interface mockup showing the three different media message view types and their expected sizes in points}

%%%%%%%%%%%%%%%%%%%%%%%%% %%%%%%%%%%%%%%%%%%%%%%%%%%
% Measuring performance % % REVIEWED | T: + | G:   %
%%%%%%%%%%%%%%%%%%%%%%%%% %%%%%%%%%%%%%%%%%%%%%%%%%%
\section{Measuring performance}
\label{sec:measuring-performance}
To gather quantifiable information and not base comparison points on visual judgement, there needed to be a way to run reproducible tests and gather numeric statistics from each test. The gathering of quantifiable information was achieved in two different parts: the creation of a repeatable test case (programmatic scrolling that simulates human interaction) and using different software to retrieve and save information about the user interface related metrics, like frame rate, and the test device's resource usage, like average CPU usage.

The following subsections describe, in depth, how the repeatable test case was created and works, how the user interface related metrics were measured, what device was used as the test device and how the device resource usage was monitored and what steps were taken to make the gathered data more easily analyzable. 

\subsection{Repeatable test case}
\label{subsec:repeatable-test-case}
The repeatable test case's purpose was to simulate human interaction through scrolling and do the exact same way every time the test is executed. The test consisted of scrolling from the top of the list view to the bottom five times, with the scroll speed continuously increasing.

To achieve the continuously increasing speed, UITableView's \mintinline{swift}{setContentOffset(contentOffset, animated: true)} function was used along with a repeating Timer. The Timer fired every 0.4 seconds and forced the view to scroll, starting from the top, towards the bottom. The scrolling was purposefully not linear: starting from 100 points, every time the Timer fired and the view was scrolled, the scroll offset value increased by 30 points. This means that the first time the Timer fired, the \mintinline{swift}{setContentOffset(contentOffset, animated: true)} was executed with a 100 point increase in the vertical content offset. The next time the Timer executed, the increase was 130 points and after that, 160 points.

To achieve the scrolling to bottom five times expectation of the test, the view's vertical offset was instantaneously reset to the 0 (causing the UITableView to start at the very top), if the inner UITableView reached the bottom and an internal timer recording the count of resets had not reached 4. If the internal timer recording the count of resets (number of scrolls to the top) reached 5, the test execution was stopped and the test was considered over.

\subsection{Measuring test case performance}
Five metrics were monitored to get an overview of the performance and efficiency of the UITableView. The measured metrics are both user interface and device resource usage related. The user interface related metrics are the test duration and frame rate. Device resource usage related metrics are average processor, graphics processor and memory usages.

Measuring of the duration of the test case and the frame rate of the user interface was done programmatically in the thesis project. Measuring the frame rates was handled by a custom implementation of the FPSCounter\cite{FPSCounterGithub} library. Measuring the test duration was done by recording the system absolute time, using the \mintinline{swift}{CACurrentMediaTime()} function, at the start of the test and comparing it against the system absolute time at the end of the test.

In addition to measuring frame rate, a custom metric called "frame rate stability" was introduced. The simplified definition of frame rate stability is "time spent within 20\% of the average frame rate". In the code, the frame rate stability was calculated as the percentage of measured frame rate points that were within 20\% of the average frame rate. Because the actual measurement technique for this metric is an estimation of its theoretical definition, its accuracy is not guaranteed to directly match its theoretical definition. For the purposes of this thesis project, the slight variation in accuracy is considered acceptable and the metric will still provide a beneficial metric value for analyzing the overall responsiveness of the user interface.

The device resource usage related metrics (average CPU, GPU and RAM usage) were measured using a third-party software called GameBench\cite{GameBenchHome}. The software required very little setup and offered very few configuration options. The test device monitoring process in GameBench was started at the same time as the repeatable scrolling test, described in \autoref{subsec:repeatable-test-case}, and stopped when the test had finished, manually. After stopping the test device monitoring process, GameBench uploaded the test data to their servers and the data became available and exportable on their website. The test data contained, among other device related information and metrics, the average CPU, GPU and RAM usage.

\subsection{Analyzing gathered metrics}
All the recorded metrics, mentioned in the subsections of this section, resulted in a total set of 6 different measurement points. The resulting list of measurement points is displayed below, in listing \autoref{lst:measurement-points}.

\begin{listing}[H]
  \caption{List of measurement points recorded for every test case}
  \label{lst:measurement-points}
  \begin{itemize}
    \item Duration (seconds);
    \item Average frame rate;
    \item Frame rate stability;
    \item Average processor usage;
    \item Average graphics processor usage;
    \item Average memory usage.
  \end{itemize}
\end{listing}

\subsection{Test device}
All tests were run on an iPod Touch 5G, as it is a good example of a low-powered device by today's standards. The 5th generation iPod touch has a retina display (resolution 1136x640)\cite{AppleIPodTouch5G} while having a weak \cite{IPhoneVsIPod5} processor (dual core Apple A5 underclocked to 800 MHz)\cite{MacObserverUnderclock}. Having a high resolution display with a weak processor means that the processor may not be able to handle complex rendering tasks that well, as the number of pixels it is required to control is too large for it to handle, among other tasks.\cite{RetinaAndWeakProcessor} The high-resolution display and weak processor combination is expected to have the effect of making the user interface responsiveness and device resource usage related metrics change more noticeably after every optimization.

%%%%%%%%%%%%%%%%%%%%%%%%%%%%%% %%%%%%%%%%%%%%%%%%%%%%%%%%
% Third party libraries used % % REVIEWED | T: + | G:   %
%%%%%%%%%%%%%%%%%%%%%%%%%%%%%% %%%%%%%%%%%%%%%%%%%%%%%%%%
\section{Third party libraries used}
Creating every component of this project from scratch would have been too large of a task to cover in this thesis, therefore third-party solutions were used to simplify some tasks. The affected tasks were for example JSON parsing, image loading and caching, detailed logging and in-code constraint creation.

The following subsections will both list all the used third-party solutions and describe their purpose in the project.

\subsection{JASON}
JASON is a faster JSON deserializer written in Swift.\cite{JASON} The sole purpose for this library in the test project was to turn the test data from JSON form into Swift objects. Even though deserialization of the JSON data could have been done using Swift's built in capabilities, JASON allowed the same goal to be achieved with much shorter and more readable code.

\subsection{SnapKit}
SnapKit is a DSL to make Auto Layout easy on both iOS and OS X.\cite{SnapKit} SnapKit was used extensively in the test project to create constraints in the code, rather than in the storyboard. Creating layout constraints in the code had many benefits over creating them in the storyboards. For example, creating programmatic constraints removed the system overhead brought in by the processing of storyboards. SnapKit also gave the engineer very precise control over each constraint and provided an overview of all constraints that exist and when they will be created. Storyboards would have only had the benefit of providing the engineer with a graphical interface in which to create those constraints.

\subsection{PINRemoteImage}
PINRemoteImage, also known as PINRemoteImageManager, is an image downloading, processing and caching manager.\cite{PINRemoteImage} PINRemoteImage was used for the exact three things brought out in the library's description: it downloaded the images, allowed them to be processed, and then stored safely in a disk cache. The PINRemoteImage library made the image downloading, processing and caching tasks require very little work to be done in the thesis project's own code. Downloading images would have been trivial with the Swift's built in tools as well, but processing and caching would have been more difficult. The ability to process the images is especially useful as it allows the images to be scaled down if necessary, which in turn makes the image smaller in dimensions and the file smaller in size. The library's cache manager is also powerful and extremely useful as it provided the option to cache different, processed, versions of the same image, including the original one. The different versions were distinguished by a string key passed to the process function.

\subsection{XCGLogger}
XCGLogger is a debug log framework for use in Swift projects.\cite{XCGLogger} XCGLogger improved Swift's built-in logging mechanism by adding a lot of useful, yet optional, information about every log statement. For example, XCGLogger's log statements displayed the thread name on which the logging function was executed on, without forcing the engineer to explicitly print out the thread name. XCGLogger also printed out function names, file names and line numbers in a similar manner to the thread names printing. The library's uses and benefits in the test project are the same as the examples brought out previously (such as the identification of threads and functions).

\subsection{FLAnimatedImageView}
FLAnimatedImage is a performant animated GIF engine for iOS.\cite{FLAnimatedImageView} FLAnimatedImageView does one thing that the iOS's built in UIImageView was not able to do: play animated images (GIFs). FLAnimatedImageView was very important to both the test project and the actual Pairby iOS application due to the fact it can display animated images.

\subsection{FPSCounter}
FPSCounter is a small library to measure the frame rate of an iOS Application.\cite{FPSCounterGithub} The FPSCounter library was used for the purpose of measuring frame rates in the test project. The data gathered by FPSCounter was used to analyze the user interface responsiveness of each optimization method (and/or test case).

%%%%%%%%%%%%%%%%%%%%% %%%%%%%%%%%%%%%%%%%%%%%%%%
% Comparison points % % REVIEWED | T:   | G:   %
%%%%%%%%%%%%%%%%%%%%% %%%%%%%%%%%%%%%%%%%%%%%%%%
\section{Comparison points}
To accurately compare performance changes in the thesis project, a set of reference points were needed and a starting point was needed to relatively compare changes to the metrics listed in listing \autoref{lst:measurement-points} after every optimization. Two reference points were created for this purpose, one called "best case scenario" and the other called "worst case scenario". Both the reference points used the same data set, which was specified in detail in \autoref{subsec:test-data-generation}. 

The following subsections describe, in detail, how the two reference points were created and what their purposes were.

\subsection{Best case scenario}
\label{subsec:best-case-scenario}
The goal of the best case scenario was to display the best possible performance. To achieve the best possible performance, simplest possible subviews were used. To create the simplest possible subviews, images were omitted from he equation, and built-in UITableViewCell class instances were used, displaying only plain text.

To remove media messages from the equation, all media messages were replaced by text messages at runtime. The text messages that replaced them contained the text "Media message with ID [N]", where N was the messageId value of the message object. Then they were rendered as text message views.

Running the performance tests yielded the results shown in \autoref{test-0-1}.
\itkIncludeImage{test-0-1}{Performance test results for the best case scenario}

As presumed before running the tests, the best case scenario test case displayed the best possible performance. The test suite was run twice and the averages of the two test cases were taken as final comparable results, as shown in \autoref{test-0-1}. The average frame rate was exactly 60 frames per second, which is the iOS golden standard\cite{IntroducingAsyncDisplayKit} as well as the upper limit\cite{WWDCFPSLimit}. Both the CPU and GPU usage remained fairly low, with the averages being 16\% and 24\% respectively. The low usages meant that neither of the processors were stressed while still providing the user with the smoothest possible experience.

\subsection{Worst case scenario}
\label{subsec:worst-case-scenario}
The goal of the worst case scenario was to display the worst possible performance. To achieve the worst possible performance, no known optimizations were applied and no common performance boosting patterns (for example reuse of cells) were used. All messages were displayed as specified in \autoref{sec:specifications}.

Displaying of cells according to the user interface specifications, listed in \autoref{sec:specifications}, required the creation of two different view classes, which both inherited from the UITableViewCell class. One of the new created classes was for displaying text messages and the other for displaying all three types (as specified in \autoref{subsec:media-message-view}) of media messages.

Running the performance tests yielded the results shown in \autoref{test-0-2}.
\itkIncludeImage{test-0-2}{Performance test results for the worst case scenario}

The test results displayed relatively worse performance compared to the best case scenario (\autoref{subsec:best-case-scenario}), as expected. The average frame rate had dropped to 26, compared to the 60 measured in the best case scenario tests. Both the processor and the graphics processor were noticeably more stressed with the average usages being 69\% and 66\% respectively, compared to the best case scenario results of 16\% and 24\%, respectively. Unexpectedly, the average memory usage for the application decreased by 3 megabytes, even though images, which use more memory both on disk and in random access memory \cite{UnderstandingFileSizes}, were used in the tests.

%%%%%%%%%%%%%%%%%%%%%%%%%%%%% %%%%%%%%%%%%%%%%%%%%%%%%%%
% Performance optimizations % % REVIEWED | T: + | G:   %
%%%%%%%%%%%%%%%%%%%%%%%%%%%%% %%%%%%%%%%%%%%%%%%%%%%%%%%
\section{Performance optimizations}
This section focuses on the practical part of the thesis, which was the creation, testing and benchmarking of different optimizations. All optimizations that had a positive effect of improving either average frame rate or frame rate stability or both were kept in the test project. This was expected to ultimately lead to the creation of a project that would have all the found optimizations with positive effects on performance.

Each of the following subsections in this section describes an optimization. Each subsection is also divided into three parts: theoretical analysis of why and how the optimization should work, benchmarking of the automated test results specified in \autoref{sec:measuring-performance} after applying the optimization, and code examples.

%%%%%%%%%%%%%%%%%% %%%%%%%%%%%%%%%%%%%%%%%%%%
% Reuse of cells % % REVIEWED | T: + | G:   %
%%%%%%%%%%%%%%%%%% %%%%%%%%%%%%%%%%%%%%%%%%%%
\subsection{Reuse of cells}
Cell reuse was chosen as the first optimization due to the fact it is one of the more widely known ones. That capability is built into the UITableView component by Apple and has been there since iOS 2.0.\cite{HackingWithSwiftCellReuse}

\subsubsection{Theoretical benefits}
Reusing of cell views means that most times, when a new message enters the view, instead of creating a new instance of the cell view class, an existing object created from that class is used. This reuse of cell views is a performance enhancement because it eliminates the overhead of cell creation.\cite{AppleCharacteristicsOfCellObjects}

This enhancement required fundamental changes to be made to the existing cell classes. Views needed to have the ability to re-adjust themselves during runtime in order match their design to the type of message they were meant to display. Previously, views only needed to define their layout state once, upon initialization, and would no longer have to readjust themselves.

\subsubsection{Performance changes}
Running the performance tests yielded the results shown in \autoref{test-1}.
\itkIncludeImage{test-1}{Performance test results after applying cell reuse optimizations}

The optimization had a positive impact on the performance of the thesis project, compared to the worst case scenario, which was the starting point. Average frame rate increased by 8.9 frames per second, while both the processor's and graphic processor's average usage decreased by 12.78 and 2.95 percent respectively. The duration of the test case also decreased from 105.5 seconds to 92.7 seconds, displaying a decrease of 12.8 seconds. The decrease in duration was directly tied to the average frame rate, as all the necessary frames were rendered faster. Average memory usage did show an increase of 3 megabytes, from 28 to 31 megabytes, but this change was small enough to classify it as minor and label it as "did not change".

\subsubsection{Code examples}
Apple has already built support for cell reuse into the UITableView component, making its implementation easy. Each of the different reusable cell classes have to be registered to the instance of the UITableView class they are meant to be reused in using the \mintinline{swift}{register(_:forCellReuseIdentifier:)} function.\cite{AppleRegisterMethod} After registering the cell classes, they can be requested from the UITableView instance by using the \mintinline{swift}{dequeueReusableCell(withIdentifier:)} function.\cite{AppleDequeueReusableCellMethod}

Listing \autoref{lst:registering-cells-code} shows an example on how to register cell classes to be reused in a specific UITableView instance.
\begin{listing}[H]
  \caption{Registering cells to be reused on a specific UITableView instance}
  \label{lst:registering-cells-code}
  \begin{minted}{swift}
    self._tableView.register(ReusingCells_TextCell.self, forCellReuseIdentifier: "ReusingCells_TextCell")
    self._tableView.register(ReusingCells_MediaCell.self, forCellReuseIdentifier: "ReusingCells_MediaCell")
  \end{minted}
\end{listing}

Listing \autoref{lst:using-registered-cells-code} shows an example on how to query reusable cells in the same UITableView instance.
\begin{listing}[H]
  \caption{Using recycled cells in the test project}
  \label{lst:using-registered-cells-code}
  \begin{minted}{swift}
    func tableView(_ tableView: UITableView, cellForRowAt indexPath: IndexPath) -> UITableViewCell {
      let message = self._messages[indexPath.row]
      let cellIdentifier = (message.mediaItems == nil) ? "ReusingCells_TextCell" : "ReusingCells_MediaCell"
      let cell = self._tableView.dequeueReusableCell(withIdentifier: cellIdentifier) as! ReusableCell
      cell.updateFromMessage(message: message)
      return cell
    }
  \end{minted}
\end{listing}

%%%%%%%%%%%%%%%%%% %%%%%%%%%%%%%%%%%%%%%%%%%%
% Static layouts % % REVIEWED | T: + | G:   %
%%%%%%%%%%%%%%%%%% %%%%%%%%%%%%%%%%%%%%%%%%%%
\subsection{Making static cell layouts}
The idea behind this possible performance optimization was to use different view classes for each of the eight different message types (see listing \autoref{lst:static-layout-cell-views}), instead of the two mutating view classes used currently. The current two view classes, text message view and media message view, adjust their layouts according to the type of message they are going to display.

\begin{listing}[H]
  \caption{List of necessary cell views to make static cell layouts possible}
  \label{lst:static-layout-cell-views}
  \begin{itemize}
    \item Inbound text message view;
    \item Outbound text message view;
    \item Inbound media item view (1 media item);
    \item Outbound media item view (1 media items);
    \item Inbound media item view (2 media items);
    \item Outbound media item view (2 media items);
    \item Inbound media item view (3 media items);
    \item Outbound media item view (3 media items).
  \end{itemize}
\end{listing}

\subsubsection{Theoretical benefits}
The theoretical performance enhancing effect of this optimization was expected to be caused by the elimination of the overhead of creating and disposing of both subviews and layout constraints. This optimization complements the previous one, as they both work towards the same goal of removing unnecessary view creations, mutations and deletions. While the previous optimization only recycled the larger parent (cell) views, this change was expected to also optimize all of the inner views and constraints.

\subsubsection{Performance changes}
Running the performance tests yielded the results shown in \autoref{test-2}.
\itkIncludeImage{test-2}{Performance test results after making cell layouts static}

The optimization had a strong effect on the average frame rate which increased by 5.7 frames per second. The frame rate stability saw an increase of 6.4 percent. The changes in the average processor, graphics processor and memory usage were so slight, they can be classified as staying the same. Overall, this optimization increased the frame rate by 16\% and frame rate stability by 6.4\% without affecting the resource usage.

\subsubsection{Code examples}
The optimization of making static cell layouts also affected the code that handled the reuse of cells. After the creation of the 8 different cell views (listed in listing \autoref{lst:static-layout-cell-views}), all of the created view classes had to be registered to the UITableView instance before they could be reused. The higher number of cell view classes also made the logic of querying for reusable cells slightly more difficult.

Listing \autoref{lst:static-cell-code-example} shows the code used in the thesis project for registering the 8 different cell views to a UITableView instance.
\begin{listing}[H]
  \caption{Registering cell views with a static layout to a UITableView instance}
  \label{lst:static-cell-code-example}
  \begin{minted}{swift}
    self._tableView.register(StaticCells_TextCell_In.self, forCellReuseIdentifier: "StaticCells_TextCell_In")
    self._tableView.register(StaticCells_TextCell_Out.self, forCellReuseIdentifier: "StaticCells_TextCell_Out")
    self._tableView.register(StaticCells_MediaCell_1_In.self, forCellReuseIdentifier: "StaticCells_MediaCell_1_In")
    self._tableView.register(StaticCells_MediaCell_1_Out.self, forCellReuseIdentifier: "StaticCells_MediaCell_1_Out")
    self._tableView.register(StaticCells_MediaCell_2_In.self, forCellReuseIdentifier: "StaticCells_MediaCell_2_In")
    self._tableView.register(StaticCells_MediaCell_2_Out.self, forCellReuseIdentifier: "StaticCells_MediaCell_2_Out")
    self._tableView.register(StaticCells_MediaCell_3_In.self, forCellReuseIdentifier: "StaticCells_MediaCell_3_In")
    self._tableView.register(StaticCells_MediaCell_3_Out.self, forCellReuseIdentifier: "StaticCells_MediaCell_3_Out")
  \end{minted}
\end{listing}

%%%%%%%%%%%%%%%%%%%%%%%%%%%%%%%%%%% %%%%%%%%%%%%%%%%%%%%%%%%%%
% Manual cell height calculations % % REVIEWED | T: + | G:   %
%%%%%%%%%%%%%%%%%%%%%%%%%%%%%%%%%%% %%%%%%%%%%%%%%%%%%%%%%%%%%
\subsection{Manual cell height calculations}
\label{subsec:manual-cell-height-calculations}
By default, the cell height calculations are done automatically by the UITableView instance's underlying logic. According to literature on this topic, these automatic calculations are often slow and it's sensible to override them with manual calculations that are more efficient.\cite{PerfectSmoothScrollingInUITableViews}\cite{HowToMakeDynamicTableViewCellHeight}\cite{MediumSmoothScrollPrearo}

\subsubsection{Theoretical benefits}
Calculating the cells heights manually using efficient calculations was expected to improve the performance when scrolling at high speeds or when the whole UITableView is refreshed using the \mintinline{swift}{reloadData()} function, which causes all cell heights to be calculated again\cite{ReloadDataHeights}.

When the calculations are done manually, it is trivial to optimize the calculations for the type of cells the application should be able to render. For example, media cells have a fixed height and calculations were not necessary in that case, as the pre-defined constant could be returned instead (after adding the defined vertical margins).

\subsubsection{Performance changes}
Running the performance tests yielded the results shown in \autoref{test-3}.
\itkIncludeImage{test-3}{Performance test results after migrating to manual cell height calculations}

The manual cell height calculations increased the average frame rate by 4.3 frames per second and the frame rate stability by 7.5 percent, while keeping the resource usage nearly the same. This optimization was considered a success as there was a 10.6\% increase in the average frame rate without affecting the resource usage. 

\subsubsection{Code examples}
To override the cell height calculations, the function \mintinline{swift}{tableView(_:heightForRowAt:)} from the UITableViewDelegate protocol has to be overwritten. An example of overriding the function, but not changing the default behavior of the UITableView is shown in listing \autoref{lst:manual-cell-height-1}.
\begin{listing}[H]
  \caption{Function for overwriting cell height calculations from the UITableViewDelegate protocol}
  \label{lst:manual-cell-height-1}
  \begin{minted}{swift}
    func tableView(_ tableView: UITableView, heightForRowAt indexPath: IndexPath) -> CGFloat {
      return UITableViewAutomaticDimension
    }
  \end{minted}
\end{listing}

Calculating the height of the media message cells is trivial as the media message views have a fixed height and fixed vertical margins. Listing \autoref{lst:manual-cell-height-2} shows how the height of the media message cells was calculated in the thesis project.
\begin{listing}[H]
  \caption{Calculating media message views' height manually}
  \label{lst:manual-cell-height-2}
  \begin{minted}{swift}
    let message = self._messages[indexPath.row]
    if let mediaItems = message.mediaItems {
      let baseHeight = (mediaItems.count == 1) ? Pairby.MessageUI.HEIGHT_MEDIA_ONE : Pairby.MessageUI.HEIGHT_MEDIA_MULTIPLE
      let verticalMarginsSum = (2 * Pairby.MessageUI.MARGIN_VERTICAL_WRAP)
      return (baseHeight + verticalMarginsSum)
    }
  \end{minted}
\end{listing}

Calculating the height of the text message cells is more complex, as the number of lines in the message content can vary. Listing \autoref{lst:manual-cell-height-3} shows how the height of the text message view is calculated, using the message text and pre-defined vertical and horizontal margins.
\begin{listing}[H]
  \caption{Calculating text cell's height manually}
  \label{lst:manual-cell-height-3}
  \begin{minted}{swift}
    let maximumWidth = (self._tableView.frame.size.width - (2 * Pairby.MessageUI.MARGIN_HORIZONTAL) - Pairby.MessageUI.MARGIN_HORIZONTAL_WRAP_DYNAMIC - Pairby.MessageUI.MARGIN_HORIZONTAL_WRAP_FIXED)
    let verticalMarginsSum = (2 * Pairby.MessageUI.MARGIN_VERTICAL + 2 * Pairby.MessageUI.MARGIN_VERTICAL_WRAP)
    let baseHeight = ceil(message.message!.heightWithConstrainedWidth(width: maximumWidth, font: Pairby.MessageUI.TEXT_FONT))
    return (baseHeight + verticalMarginsSum)
  \end{minted}
\end{listing}

%%%%%%%%%%%%%%%%%%%%%%% %%%%%%%%%%%%%%%%%%%%%%%%%%
% Cell height caching % % REVIEWED | T: + | G:   %
%%%%%%%%%%%%%%%%%%%%%%% %%%%%%%%%%%%%%%%%%%%%%%%%%
\subsection{Cell height caching}
The previous optimization of using efficient manual cell height calculations resulted in a noticeable increase in performance. Reducing the number of calculations altogether should therefore also result in a boost in performance. Reducing the number of calculations was expected to be possible by caching the calculation results in memory and only doing the calculations if no cached value could be found.

\subsubsection{Theoretical benefits}
Caching the calculation results was expected to reduce the number of calculations, and through that strain the processor less, when the user scrolls the same content multiple times. This optimization would not provide any performance benefits the first time the user opens or scrolls the list view, as the in-memory cache of the calculated cell heights would be empty. When the user reaches content they have already seen, the calculation should theoretically not be run and the cached value used instead.

\subsubsection{Performance changes}
Running the performance tests yielded the results shown in \autoref{test-4}.
\itkIncludeImage{test-4}{Performance test results after the implementation of caching manually calculated cell heights}

As expected, the performance did improve. The average frame rate increased by 3.9 frames per second and the frame rate stability by 6.95\%. The average processor, graphics processor and memory usages, similarly to previous tests, remained nearly the same. Average device resource usages stayed nearly the same. The implementation of caching manually calculated cell heights improved the user interface responsiveness while not affecting the resource usages.

\subsubsection{Code examples}
The cache was a Dictionary, in which the key was the cell's row index and the value was the height as a CGFloat. Listing \autoref{lst:caching-1} shows an example of the whole cycle of calculating the cell heights manually, storing and using the cached value. The manual calculations have been omitted from this listing, they can be found in \autoref{subsec:manual-cell-height-calculations}. 
\begin{listing}[H]
  \caption{Using a Dictionary to store and use manually calculated cell heights for caching purposes}
  \label{lst:caching-1}
  \begin{minted}{swift}
    private var _heightCache: [Int: CGFloat] = [:]

    func tableView(_ tableView: UITableView, heightForRowAt indexPath: IndexPath) -> CGFloat {
        if let height = self._heightCache[indexPath.row] { return height }
        let message = self._messages[indexPath.row]

        if let mediaItems = message.mediaItems {
            let Height = (mediaItems.count == 1) ? Pairby.MessageUI.HEIGHT_MEDIA_ONE : Pairby.MessageUI.HEIGHT_MEDIA_MULTIPLE
            let Margins = (2 * Pairby.MessageUI.MARGIN_VERTICAL_WRAP)
            self._heightCache[indexPath.row] = (Height + Margins)
            return self._heightCache[indexPath.row]!
        }

        let MaxWidth = (self._tableView.frame.size.width - (2 * Pairby.MessageUI.MARGIN_HORIZONTAL) - Pairby.MessageUI.MARGIN_HORIZONTAL_WRAP_DYNAMIC - Pairby.MessageUI.MARGIN_HORIZONTAL_WRAP_FIXED)
        let Margins = (2 * Pairby.MessageUI.MARGIN_VERTICAL + 2 * Pairby.MessageUI.MARGIN_VERTICAL_WRAP)
        let Height = ceil(message.message!.heightWithConstrainedWidth(width: MaxWidth, font: Pairby.MessageUI.TEXT_FONT))
        self._heightCache[indexPath.row] = (Height + Margins)
        return self._heightCache[indexPath.row]!
    }
  \end{minted}
\end{listing}


%%%%%%%%%%%%%%%%%%%%%%%%%% %%%%%%%%%%%%%%%%%%%%%%%%%%
% Optimizing image sizes % % REVIEWED | T: + | G:   %
%%%%%%%%%%%%%%%%%%%%%%%%%% %%%%%%%%%%%%%%%%%%%%%%%%%%
\subsection{Optimizing image sizes}
\label{sec:optimizing-image-sizes}
As specified in the specification section, the test data contained about 25\% of media messages, all of which contained 1 to 3 images (both static and animated). Since images make up a relatively large portion of all messages in the test data set, it was decided to look into optimizing them by resizing them to the correct size before displaying them in the user interface. The idea was a collaboration between the author and the thesis' supervisor Gary Planthaber.

PINRemoteImage already provided the functionality to do image processing and caching of the processed images, but that was only possible for non-animated images. It was decided to not optimize the animated images optimized and let the system handle them as they are.

\subsubsection{Theoretical benefits}
The pre-processing should make the images the correct size before they reach the user interface and if the static images would no longer need any up- or downscaling in terms of processing, this should theoretically improve the rendering times of the media cells and the performance when the scrollable view is in motion.

\subsubsection{Performance changes}
Running the performance tests yielded the following results:
\itkIncludeImage{test-5}{Performance test results after adding pre-processing of images}

The change in the average frame rate did not display a big change (improvement of 1.7 frames per second), but the frame rate stability increased by 30.5\%, which was a huge improvement. The processor and graphics processor usage did not display a noticeable change. The memory usage, however, changed dramatically. The average memory usage increased by 136\%, from 33 megabytes to 78 megabytes, compared to the previous test. This average memory increase is easily explainable by the fact that the image processing uses a lot of memory and the application had to start a lot of image processing processes when the scroll speed was nearing its maximum.

The average memory usage does seem high, but since it is temporary for the duration of the fast scroll, it should definitely be worth the huge improvements in the frame rate stability.

\subsubsection{Code examples}
Below is an example of how the image was resized to a given size, using iOS' built-in classes. The parameters of the function were the original UIImage and the expected size of the new image. Since the handled images are always square, only one dimension's length was enough to specify the size.
\begin{listing}[H]
  \caption{Resizing an UIImage to a specified size}
  \begin{minted}{swift}
    func processImage(image: UIImage, size: CGFloat) -> UIImage {
      let scaledSize = (size * UIScreen.main.scale)
      let imageRect = CGRect(x: 0, y: 0, width: scaledSize, height: scaledSize)

      UIGraphicsBeginImageContext(imageRect.size)

      let sizeMultiplier: CGFloat = (scaledSize / image.size.width)

      var drawRect = CGRect(x: 0, y: 0, width: image.size.width * sizeMultiplier, height: image.size.height * sizeMultiplier)
      if (drawRect.maxX > imageRect.maxX) { drawRect.origin.x -= (drawRect.maxX - imageRect.maxX) / 2 }
      if (drawRect.maxY > imageRect.maxY) { drawRect.origin.y -= (drawRect.maxY - imageRect.maxY) / 2 }

      image.draw(in: drawRect)

      let processedImage = UIGraphicsGetImageFromCurrentImageContext()
      UIGraphicsEndImageContext()

      return processedImage!
    }
  \end{minted}
\end{listing}


%%%%%%%%%%%%%%%%%%%%%%% %%%%%%%%%%%%%%%%%%%%%%%%%%
% Layer rasterization % % REVIEWED | T: + | G:   %
%%%%%%%%%%%%%%%%%%%%%%% %%%%%%%%%%%%%%%%%%%%%%%%%%
\subsection{Rasterization of cell layers}
Rasterization in the iOS view rendering context means turning the view into a cacheable bitmap that can be used in later rendering more efficiently. Rasterization of views was a built in feature of the CALayer class. Every view has a CALayer instance, accessible through the \mintinline{swift}{layer} property. That property is inherited from the UIView base class.

\subsubsection{Theoretical benefits}
All the views except the ones containing animated images are static once rendered and will not change in the UI. This was one of the requirements to reap benefits from layer rasterization. Rasterization process consumes more resources and time than regular rendering at first, but once done, the view would be handled much more efficiently by the rendering engine due to the fact that the graphics processor can use the cached bitmap instead of re-drawing the view.\cite{MovingPixelsOntoTheScreen}

\subsubsection{Performance changes}
Running the performance tests yielded the following results:
\itkIncludeImage{test-6}{Performance test results after enabling rasterization of the views}

Cell rasterization did not provide any of the expected performance improvements. The average frame rate decreased by 2.8 frames per second and the frame rate stability went down by almost 30\%.

This shocking result confirmed that more research needed to be done about the proper usage of rasterization. These benefits are investigated in the next section.

\subsubsection{Code examples}
Enabling the rasterization of the cell views required adding only two lines of code. There is an example of the two lines below. The first one enabled rasterization and the second one told the rasterization mechanism to use a certain scale to compensate for the fact that some devices are using retina screens and require scaling when being rasterized.
\begin{listing}[H]
  \caption{Rasterizing views}
  \begin{minted}{swift}
    self.view.layer.shouldRasterize = true
    self.view.layer.rasterizationScale = UIScreen.main.scale
  \end{minted}
\end{listing}

%%%%%%%%%%%%%%%%%%%%%%%%%%%%%%%% %%%%%%%%%%%%%%%%%%%%%%%%%%
% Avoiding offscreen rendering % % REVIEWED | T: + | G:   %
%%%%%%%%%%%%%%%%%%%%%%%%%%%%%%%% %%%%%%%%%%%%%%%%%%%%%%%%%%
\subsection{Avoiding unnecessary offscreen rendering}
\label{sec:avoiding-unnecessary-offscreen-rendering}
Offscreen rendering means drawing the view into a new buffer (bitmap cache) which is offscreen (not on the screen) before drawing that buffer on the screen. The conventional way of drawing views is to draw them subview by subview directly onto the screen.\cite{MovingPixelsOntoTheScreen}

Offscreen rendering consumes more resources and is slower than conventional rendering, but it can be beneficial to performance due to the fact that the resulting bitmap can be cached and reused. The caching and reusing will only work and improve performance when the view does not change that often. The ideal case would be a view that never changes (is immutable). If a view were to change often, for example a view containing an animated image (GIF), then the drawing to bitmap would have had to be done every time the animated image's frame changed and none of the cached buffers could be reused.

Layer rasterization was one example of manually triggering offscreen rendering on a whole view. Offscreen rendering can also be triggered automatically by Core Animation. This would happen for example when a mask is directly or indirectly applied to a layer. That would also force the graphics processor to do offscreen rendering in order to apply that mask. This will in turn, as stated earlier, put unnecessary burden on the GPU.\cite{MovingPixelsOntoTheScreen}

In order to get benefits from the rasterization of layers, all other direct or indirect causes of offscreen rendering should be avoided. This is theoretically why the previous optimization did not work - some indirect causes of offscreen rendering were invalidating the rasterization cached bitmap. For example, some indirect causes of offscreen rendering are applying shadows and corner radiuses to the layer, the latter of which the test project does for all cells to create the chat bubble effect.\cite{MovingPixelsOntoTheScreen}

\subsubsection{Theoretical benefits}
The theoretical benefits were reduced resource usage and increased performance (better average frame rate, higher frame rate stability). These should have also resulted in smoother scrolling.

\subsubsection{Performance changes}
Running the performance tests yielded the following results:
\itkIncludeImage{test-7}{Performance test results after removing unnecessary offscreen rendering}

Avoiding unnecessary offscreen rendering proved to be beneficial to performance and graphics processor usage. The average frame rate increased by 3.1 and the stability increased by almost 10\%. While the average processor usage did not change much, the average graphics processor usage showed a huge drop of 29\%. The average memory usage, however, increased by a whopping 30\% (from 78 megabytes to 101.5 megabytes). 

\subsubsection{Code examples}
In order to work around using CALayer's corner radius when drawing text cell views, the background with a corner radius was drawn manually. This was done by creating a custom wrapper view for the text that was able draw the rounded corner background efficiently. Below is the code for that view, which shows the background color being customizable and the draw method being only aware of drawing a rectangle with rounded corners.
\begin{listing}[H]
  \caption{Custom wrapper view for text message view with rounded corners that avoids offscreen rendering}
  \begin{minted}{swift}
    class NoOffscreenRendering_TextCell_LabelWrapView: UIView {
      private var _fillColor: UIColor?

      override init(frame: CGRect) {
        super.init(frame: frame)
        self.backgroundColor = Pairby.Colors.GRAY_BG
      }

      convenience init() {
        self.init(frame: CGRect.zero)
      }

      required init(coder aDecoder: NSCoder) {
        fatalError("This class does not support NSCoding")
      }

      func fillWith(color: UIColor) {
        self._fillColor = color
        self.setNeedsDisplay()
      }

      override func draw(_ rect: CGRect) {
        UIBezierPath(roundedRect: rect, cornerRadius: 15.0).addClip()
        self._fillColor?.setFill()
        UIBezierPath(rect: rect).fill()
      }
    }
  \end{minted}
\end{listing}


%%%%%%%%%%%%%%%%%%% %%%%%%%%%%%%%%%%%%%%%%%%%%
% AsyncDisplayKit % % REVIEWED | T: + | G:   %
%%%%%%%%%%%%%%%%%%% %%%%%%%%%%%%%%%%%%%%%%%%%%
\subsection{AsyncDisplayKit}
AsyncDisplayKit is an iOS framework built on top of UIKit that keeps even the most complex user interfaces smooth and responsive.\cite{IntroducingAsyncDisplayKit} The library's goal was to make even the most complex views have a 60 frames per second rendering in iOS applications by moving the rendering off the main thread.

In order to test this out, a replica of the best case scenario view was built, identical to the one built with UIKit. This meant only having text message views. Media messages were replaced with text. Instead of UIKit classes, AsyncDisplayKit's classes and methodologies were used.

\subsubsection{Theoretical benefits}
The expected test results after applying AsyncDisplayKit's classes and methodologies was performance at least equivalent to that of UIKit's best case scenario. The expected benefits would be less resources consumed due to the better use of threads.

\subsubsection{Performance changes}
Running the performance tests yielded the following results:
\itkIncludeImage{test-8}{Performance test results with AsyncDisplayKit's classes and methodologies}

Not only was the performance noticeably worse compared to the UIKit's best case scenario, the time it took to open the view initially was unacceptable (more than 10 seconds to load all the AsyncDisplayKit's classes and views). UIKit's best case scenario view opened instantly, with no main thread holdups.

Scrolling performance wise, it performed much worse than UIKit's UITableView. The average frame rate did not meet the 60 frames per second expectation. The frame rate stability was also not a solid 100 percent. The actual recorded values were 58.5 frames per second and 97.5\% respectively. CPU usage was 26\% higher than that of UIKit's best case scenario. Graphics processor and memory usage did not change notably.

Due to the best case scenario results being so poor, it was decided not to continue trying to make AsyncDisplayKit work as an optimization.

\subsubsection{Code examples}
Below is a code example of how cells were created and returned to be used in the user interface using AsyncDisplayKit's classes and functions.
\begin{listing}[H]
  \caption{Creation and use of AsyncDisplayKit's cell views}
  \begin{minted}{swift}
    func tableNode(_ tableNode: ASTableNode, nodeBlockForRowAt indexPath: IndexPath) -> ASCellNodeBlock {
      let message = self._messages[indexPath.row]
      return {
        let cell = ASTextCellNode()
        cell.text = (message.message != nil) ? message.message! : "Media message with ID [\(message.id)]"
        return cell
      }
    }
  \end{minted}
\end{listing}

%%%%%%%%%%%%%%%%%%%%%%%%%% %%%%%%%%%%%%%%%%%%%%%%%%%%
% End result comparisons % % REVIEWED | T:   | G:   %
%%%%%%%%%%%%%%%%%%%%%%%%%% %%%%%%%%%%%%%%%%%%%%%%%%%%
\section{Findings}
The end result was considered to be the result of applying all the optimizations cumulatively from recycling cells to avoiding unnecessary offscreen rendering. The optimizations included reusing cells, using static views, manual height calculations, height caching, image optimizations and avoiding unnecessary offscreen rendering. This meant considering the end result to be the product of \autoref{sec:avoiding-unnecessary-offscreen-rendering}.

\subsection{Performance relative to worst case scenario}
To better compare the end result against the worst case scenario, a table comparing the worst case scenario results from \autoref{subsec:worst-case-scenario} against the results from \autoref{sec:avoiding-unnecessary-offscreen-rendering} was created and is displayed below.
\itkIncludeImage{worst-vs-end}{Worst case scenario results compared against the end result's test results}

The duration of the test decreased from 105.5 seconds to 87.4, which is a 17\% decrease. This can be explained by the increased performance (better frame rate), as the processors could finish rendering all the required frames faster.

The average frame rate increased from 26.0 to 53.4, which is a tremendous 105\% increase, effectively doubling the number of frames each second. The cap and goal of 60 frames per second was not achieved.

The frame rate stability index improved more than 2.6 times, from the starting value of 23.6\% to 87.1\%. This custom measurement point effectively showed that not only did the average frame rate increase, the time spent within 20 percent of that average also increased by a big margin.

Average processor and graphics processor both decreased by a noticeable margin. Average processor usage went from 69.4\% to 57.8\%, which is a 11\% decrease (comparative decrease of 16\%). Graphics processor's average usage decreased from 65.7\% to 40.1\% (comparative decrease of 38.9\%). Processors' usages went down while performance improved, which was the goal.

Average memory usage increased by 2.6 times, from an average of 28 megabytes to 101.5 megabytes. The main reason for this increase was image processing. The biggest increase in average memory usage can be traced to \autoref{sec:optimizing-image-sizes}, where image optimizations were introduced. The mentioned optimizations increased average frame rate and frame rate stability while sacrificing memory usage. The memory usage increase is temporary, while the processing is first done and after properly caching the processed images, the memory usage will decrease.

To conclude the comparison between the worst case scenario and the end result, it can be said that the optimizations were a success. The average frame rate more than doubled and the frame rate stability almost quadrupled. Average usages of both the graphics processor and the central processor decreased. Memory usage did increase by a large margin, but as the increase is temporary and will not affect the application's memory usage for the whole duration of the application's life, its importance can be disregarded. The optimizations can be called useful, when comparing them against the worst case scenario, as the performance improved while resource usage decreased.

\subsection{Performance relative to best case scenario}
To compare the end result against the goal, which was the best case scenario, another table was created. The table displayed the differences between the test results from \autoref{subsec:best-case-scenario} against the results from \autoref{sec:avoiding-unnecessary-offscreen-rendering}. The created table, which also contains comparative percentages, is displayed below.
\itkIncludeImage{best-vs-end}{Best case scenario results compared against the end result's test results}

The duration of the end result was still 21 seconds longer than that of best case scenario's. As was the case with the worst case scenario, this can be explained by the changes in performance. When the necessary frames are rendered faster, the duration will be less. This in turn is caused by the fact that the test ends when the scroll reaches the bottom.

The best case scenario's average frame rate displayed that the cap is 60 frames per second and it is possible to achieve it with views that have very little complexity. The end result did not achieve the same average frame rate. Instead, the tests recorded the average frame rate to be 53.4. Even though it was proven that 60 frames per second can be achieved, the end result with all known optimizations could not reach it.

The frame rate stability for the best case scenario proved the possibility of having a stability index of 100 percent. The end result did not manage to reach the 100 percent. Instead, the tests recorded a result of 87.1\%. As explained previously, this means that 87\% of the time was spent within 20 percent of the average frame rate, which was 53.4 frames per second. The 87.1\% stability was still an acceptably good result.

Comparing the average central processor and graphics processor usage's in the best case scenario and end product's test results showed a shocking difference. Average processor usage was only 16\% in the best case scenario, but 57.8\% in the end result, which is about 3.5 times higher. The change was less drastic for the average graphics processor usage which increased from 23.8\% to 40\%. These changes can be explained by the fact that the best case scenario did not have any complex views nor images in it. The processors' average usages did increase by a noticeable amount, but still stayed within reasonable bounds and did not come close to their upper limit.

The average memory usage for the best case scenario was about the same as for all the optimizations until \autoref{sec:optimizing-image-sizes}, where image optimizations were introduced. After that point, the memory usage increased and the effect was worsened by the additional image processing introduced in \autoref{sec:avoiding-unnecessary-offscreen-rendering}. The average memory usage hovered near 101 megabytes after these optimizations, compared to the original 31 megabytes, as recorded in the best case scenario. As also mentioned in the previous subsection, end result compared to the worst case scenario, the memory usage would not stay at that recorded point, it would eventually settle down back to 31 megabytes when all the image processing is done. The average memory usage increased more than 3 times, but due to the fact that it is a temporary state, it is not considered a deal breaker.

%%%%%%%%%%% %%%%%%%%%%%%%%%%%%%%%%%%%%
% Summary % % REVIEWED | T: + | G:   %
%%%%%%%%%%% %%%%%%%%%%%%%%%%%%%%%%%%%%
\iosection{Summary}
The goal of the thesis was to improve the Pairby iOS application's message list view's UITableView's performance. In total, 8 different optimization methods were found, realized and benchmarked. 6 of the found optimizations had the expected performance boosting effect. 2 of the optimizations did not provide the expected performance enhancing benefits, but were still included in this thesis since their failure reasons became a basis for the next optimizations.

The practical part of the thesis started by creating two benchmarking points called the worst case scenario and the best case scenario. Their goal was to represent the worst possible performance and the best possible performance. The worst case scenario with its  view became the starting point for all optimizations while the 60 frame per second performance recorded in best case scenario view became the ultimate goal.

The whole optimization process was split into sections, each covering one specific optimization technique. Each section had three subsections in addition to the short introduction: theoretical benefits, performance changes and code examples. This structure allowed each technique to be explained in adequate depth in a similar manner across all sections.

Even though the optimizations were handled individually per section, all optimizations were applied accumulatively to the main project throughout the optimization process. The purpose of that was to ultimately create a view which had all the optimizations and eventually mark it as the end result.

To draw conclusions from done work, the end result, which was a result of all found and applied optimizations, was compared to the worst and best case scenarios.

Compared to the worst case scenario, the improvements were huge. The average frame rate increased by 105\% and ended up at a measured value of 53.4 frames per second. The frame rate stability (time spent within 20\% of the average frame rate) increased from 23\% to 87\%, which is a huge improvement. The average central processor and graphics processor usages decreased by 11.5\% and 25.6\% respectively. The respective values for the average CPU and GPU usages measured for the end result were 57.9\% and 40.2\%. The average memory usage showed a huge increase from 28.0 megabytes to 101.5 megabytes. That increase can be explained by the addition of image processing, which improved average frame rates and frame rate stability at the cost of temporarily using more memory. When comparing the end result to the worst case scenario, all the performance related parameters increased by more than two times while the average CPU and GPU usage went down and the memory usage showed a temporary increase.

The end result did not achieve the performance numbers measured and set as a goal in the best case scenario tests. The best case scenario showed that an average frame rate 60 frames per second and a frame rate stability of 100\% is the best possible performance. The end result benchmarks recorded an average frame rate of 53.4 frames per second and a frame rate stability of 87.1\%. Both of those metrics came close to 90\% of the goal, which can be counted as a success. The average CPU, GPU and memory usages increased by a lot. Average CPU usage increased from 16.1\% to 57.9\%, GPU usage from 23.9\% to 40.2\% and memory usage from 31.0 to 101.5 megabytes. When comparing the end result to the set goal, about 90\% of the performance goal was achieved, which can be counted as a success.

The author finds that the user experience provided by the end result is great. The end result offered about 90\% of the best possible performance and the author finds that a frame rate of 53.4 frames per second is indistinguishable from 60 frames per second by an average user. This means that the goal of the thesis, which was to offer a good user experience related to scroll performance, was achieved.

%%%%%%%%%%%%%%%%%%%%%%%%%%%%%%%%
%%%% After-content sections %%%%
%%%%%%%%%%%%%%%%%%%%%%%%%%%%%%%%

%%%%%%%%%%%
% Figures %
%%%%%%%%%%%
\newpage
\phantomsection
\addcontentsline{toc}{section}{List of Figures}
\listoffigures

%%%%%%%%%%%%%%
% References %
%%%%%%%%%%%%%%
\newpage
\phantomsection
\addcontentsline{toc}{section}{References}
\bibliography{thesis}
\bibliographystyle{plain}

\end{document}
